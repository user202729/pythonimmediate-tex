%! TEX program = pdflatex
\documentclass[a5paper]{article}
\errorcontextlines=5
\usepackage{prettytok}
\prettyinitterm

\ifdefined\prettyinittermcat
	\prettyinittermcat
\fi

\ExplSyntaxOn
\pretty:n{going to load}

\tl_set:Nn\tmp{\usepackage[}

\def\tmparg{~--sanity-check-extra-line~}
%\def\tmparg{~ --debug ~ 9 ~}
\ifdefined\specifymode
	\tl_put_right:Nx\tmparg{~ --mode ~ \specifymode}
\fi
\ifdefined\testnaiveflush
	\tl_put_right:Nx\tmparg{~ --debug-force-buffered ~ --naive-flush}
\fi
\ifdefined\extraarg
	\tl_put_right:Nx\tmparg{~\extraarg}
\fi
\tl_put_right:Nx\tmp{args={\tmparg},}

\tl_put_right:Nn\tmp{python-executable={PYTHONPATH=/home/user202729/TeX/pythonimmediate/ ~ python3.8},}
\tl_put_right:Nn\tmp{]{pythonimmediate}}

\ExplSyntaxOff
\tmp





\ExplSyntaxOn
\pretty:n{1.start}

\begin{pycode}
import pythonimmediate

x="xxx"
import sys
print("======== test stdout ========")
print("======== test stderr ========", file=sys.stderr)

#print=pythonimmediate.print
printT=pythonimmediate.print_TeX
pythonimmediate.run_tokenized_line_local(r'\pretty:n {2.a}')
\end{pycode}

\pycv|pythonimmediate.run_block_local(r'\pretty:n {3.xxx~ =~ ' + x + '}')|

\begin{pycode}
pythonimmediate.run_block_local(r'''
\pretty:n {4.}
\pycv|pythonimmediate.run_tokenized_line_local(r'\pretty:n{5.}')|
\pretty:n {6.}
''')
\end{pycode}

\begin{pycode}

x=123

pythonimmediate.run_tokenized_line_peek(
r'\pretty:n{7.} \pyc{x=456} \pretty:n{8.} \pythonimmediatecontinue {}'
)

assert x==456

content=pythonimmediate.run_tokenized_line_peek(r'\pythonimmediatecontinue {abc~~def}')
assert "abc def"==content, content

pythonimmediate.run_tokenized_line_local(
r'\pretty:n{9.}'
)
\end{pycode}

\pretty:n{10.}


\ExplSyntaxOff


\begin{document}




1+1=\py{1+1}.

\pyc{x=3}

x=\py{x}.

% \pyc{printT("%")}  % ← error!

x=\pyc{printT(str(x), end="")}.  % without trailing space. Note that this requires % to have the "comment" catcode.

x=\pyc{printT(str(x))}.  % there's an extra space.

\begin{pycode}
print("11.\n")
\end{pycode}

% ======== test empty code block
\begin{pycode}
\end{pycode}


\begin{pycode}

import unittest

class Test(unittest.TestCase):
	def test_preserve_spaces(self)->None:
		self.assertEqual("""
  a 
""".count(' '), 3)

	def test_preserve_tabs(self)->None:
		self.assertEqual("""
		a			
""".count('\t'), 5)

	def test_preserve_unicode_chars(self)->None:
		for char, code in zip(
					   'Ʋ×⁴ℝ𝕏',
					   [198, 178, 215, 8308, 8477, 120143]
					   ):
			assert ord(char)==code

suite = unittest.defaultTestLoader.loadTestsFromTestCase(Test)
result = unittest.TextTestRunner(failfast=True).run(suite)
assert not result.errors


x=y=z=1
\end{pycode}

\def\testFilePath{helper_file.py}
\def\letYThree{y=3}
\def\letZFour{z=4}

% ======== test indirect execution
\pyfile{\testFilePath}  % x=2
\pyc{\letYThree}
\pycq{\letZFour}

\begin{pycode}
assert x==2
assert y==3
assert z==4
x=y=z=1
\end{pycode}

% ======== test direct execution
\pyc{y=3}
\pycq{z=4}

\begin{pycode}
assert y==3
assert z==4
x=y=z=1
\end{pycode}

% ======== test more symbols
\pyc{assert ord(r' ')   ==0x20}
\pyc{assert ord(r'\ ')  ==0x20}
\pyc{assert len(r'  ')  ==1}  % unfortunate
\pyc{assert len(r'\ \ ')==2}
\pyc{assert ord('\\\\')==0x5c}
\pyc{assert ord(r'\{')  ==0x7b}
\pyc{assert ord(r'\}')  ==0x7d}
\pyc{assert ord(r'\$')  ==0x24}
\pyc{assert ord(r'$')   ==0x24}
\pyc{assert ord(r'\&')  ==0x26}
\pyc{assert ord(r'&')   ==0x26}
\pyc{assert len(r'#')   ==2}  % unfortunate
\pyc{assert len(r'\#')  ==1}
\pyc{assert ord(r'\#')  ==0x23}
\pyc{assert ord(r'^')   ==0x5e}
\pyc{assert ord(r'\^')  ==0x5e}
\pyc{assert ord(r'_')   ==0x5f}
\pyc{assert ord(r'\_')  ==0x5f}
\pyc{assert ord(r'\%')  ==0x25}
\pyc{assert ord(r'~')   ==0x7e}
\pyc{assert ord(r'\~')  ==0x7e}

\end{document}
